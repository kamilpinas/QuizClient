\documentclass[a4paper]{article}
\usepackage[a4paper, portrait, margin=2cm]{geometry}
\usepackage{polski}
\usepackage[utf8]{inputenc}
\newcounter{para}
\newcommand\mypara{\par \refstepcounter {para} \huge \textbf \thepara.\space}\usepackage[T1]{fontenc}
\usepackage{inconsolata}
\usepackage{graphicx}
\graphicspath{ {./images/} }
\usepackage{color}
\definecolor{pblue}{rgb}{0.13,0.13,1}
\definecolor{pgreen}{rgb}{0,0.5,0}
\definecolor{pred}{rgb}{0.9,0,0}
\definecolor{pgrey}{rgb}{0.46,0.45,0.48}
\usepackage{listings}
\lstset{language=Java,
  showspaces=false,
  showtabs=false,
  breaklines=true,
  showstringspaces=false,
  breakatwhitespace=true,
  commentstyle=\color{pgreen},
  keywordstyle=\color{pblue},
  stringstyle=\color{pred},
  basicstyle=\ttfamily,
  moredelim=[il][\textcolor{pgrey}]{$$},
  moredelim=[is][\textcolor{pgrey}]{\%\%}{\%\%}
}


\begin{document}
%POCZATEK STRONY TYTULOWEJ

\begin{center}
\title*{\large Narzędzia i środowiska programistyczne}\\
\end{center} 

\noindent\rule{\textwidth}{1pt}

\vspace{40px}

\begin{center}
\section*{\Huge Dokumentacja projektu Quiz}
\end{center} 
\vspace{180px}


\begin{center}
\author*{\Large   Wykonali: \\ Gabriela Jasnosz\\ Kamil Budzik\\ Kamil Pinas\par}
\end{center} 

\begin{flushleft}
\vspace{260px}
\author*{\Large  Prowadzący: Tomasz Gądek \newline Kierunek: Informatyka \newline Semestr: Zimowy \newline  Rok: 2\par}
\end{flushleft}
\newpage
%KONIEC STRONY TYTUŁOWEJ

%WSTEP
\setlength\parindent{0pt}
\mypara Wstęp
\section*{\large\hspace*{8mm} Stworzona przez nas aplikacja jest Quizem działającym w oparciu o architekturę klient-serwer. Server odczytuje w całości plik csv z pytaniami, po czym kolejno przesyła je do klienta na jego żądanie.
\newline \newline Klient: Zajmuje się jedynie nawigacją i wyświetlaniem pytań wraz z odpowiedziami wielokrotnego wyboru. Klient wyświetla również ilość możliwych punktów do zdobycia za każde pytanie.
\newline \newline Serwer: Jest odpowiedzialny za zwracanie odpowiednich pytań w formacie JSON na żądanie klienta. Przelicza na bieżąco punkty. Po zakończeniu testu serwer generuje raport z wynikami w formacie PDF. W raporcie znajduje się informacja jakie odpowiedzi zaznaczył użytkownik i jakie odpowiedzi były prawidłowe.W podsumowaniu znajduje się również informacja o wyniku testy - czy został zdany i ile punktów uzyskał zdający.}\par
%/WSTEP

%INSTRUKCJA OBSLUGI
\mypara Instrukcja obsługi i opis działania aplikacji klienta
\section*{\large\hspace*{8mm} 1.Kliknij przycisk “Start Game” aby rozpocząć quiz.
Po kliknięciu przycisku uruchamia się usługa odpowiedzialna za odczytywanie pytań z serwera kolejno, zgodnie z id pytania.}\par

%startgamephoteo

\begin{center}
\includegraphics[width=12cm]{start.png}
\end{center}
\newpage %KONIEC STRONY

\section*{\large\hspace*{8mm} 2.Zaznacz odpowiedzi, które uważasz za poprawne.}\par
\begin{center}
\includegraphics[width=12cm]{wybierzodp.png}
\end{center}
\section*{\large\hspace*{8mm} Po wybraniu odpowiedzi odblokowuje się przycisk Next question, którym możesz przejść do następnego pytania. Kliknięcie przycisku automatycznie spowoduje uruchomienie usługi serwera odpowiedzialnej za zliczanie punktów na bieżąco.
UWAGA! Tylko wybranie wszystkich poprawnych odpowiedzi spowoduje przyznanie punktów za pytanie.}\par
\section*{\large\hspace*{8mm} 3.Gdy odpowiesz na wszystkie pytania kliknij przycisk Send result and quit, który spowoduje wyslanie do serwera wszystkich informacji o przeprowadzonym quizie oraz uruchomienie uslugi odpowiedzialnej za utworzeniu raportu z przeprowadzonego quizu w formacie pdf.}\par
\begin{center}
\includegraphics[width=12cm]{koniec.png}
\end{center}

\section*{\large\hspace*{8mm} 4.Potwierdź swoją decyzję, aby zamknąc aplikację.}\par
\begin{center}
\includegraphics[width=12cm]{potweirdz.png}
\end{center}
%/INSTRUKCJA OBSLUGI

%OPIS USLUG
\mypara Opis usług serwera
%%GET POCZTEK
\section*{\large\hspace*{8mm} a) Usługa get}\par
\small
\begin{lstlisting}
   public ResponseEntity<ReturnQuestion>test(@PathVariable("id") Integer id) {
        try {
            boolean ifLast;
            if (id == quiz.size()-1) {
                ifLast = true;
            } else {
                ifLast = false;
            }
            final ReturnQuestion quizValues = new ReturnQuestion(
                    quiz.get(id).getQuestion(),
                    quiz.get(id).getAnswerA(),
                    quiz.get(id).getAnswerB(),
                    quiz.get(id).getAnswerC(),
                    quiz.get(id).getAnswerD(),
                    quiz.get(id).getPoints(),
                    ifLast);
            return new ResponseEntity<ReturnQuestion>(quizValues, HttpStatus.OK);
        }catch(Exception e){
            return new ResponseEntity<>(HttpStatus.NOT_FOUND);
        }
\end{lstlisting}
\section*{\large\hspace*{8mm}Parametry wejściowe: id(numer pytania)\newline
\hspace*{8mm}Parametry wyjsciowe (JSON):}\par
\small
\begin{lstlisting}
    {
        "question": "2+2=?",
        "answers": ["0","1","2","4"],
        "points": 1,
        "lastQuestion": false
    }
\end{lstlisting}

\section*{\large\hspace*{8mm} Usługa get przyjmuje parametr którym jest numer pytania. Następnie tworzy obiekt klasy ReturnQuestion, który przechowuje wszystkie wymagane informacje odpowiadające pytaniu na liscie o podanym id, które mają być przesłane do klienta, czyli: pytanie, tablicę z czterema odpowiedziami, liczbę punktów za pytanie oraz informacje o tym czy pytanie jest ostatnie. 
Jeżeli wszystko zadziała poprawnie tj. jeżeli istnieje pytanie o danym id oraz plik csv został odczytany prawidłowo, kod odpowiedzi serwera wyniesie 200(OK), a obiekt quizValues zostanie przesłany w formacie JSON.W przeciwnym razie odpowiedzią serwera będzie kod 404(NOTFOUND)\newline
}\par
%% GET KONIEC

%PUT POCZATEK
\section*{\large\hspace*{8mm} b) Usługa PUT URL(/quiz/calculate)
}\par
\small
\begin{lstlisting}
    @PutMapping(value = "/quiz/calculate")
    public ResponseEntity<AnswerDto> test2(@RequestBody AnswerDto answerDto) {
    
        int plus=QuizCode.checkAnswer(answerDto.getSelectedAnswers(),
        quiz.get(answerDto.getQuestionId()) .getCorrectAnswers(),
        Integer.parseInt(quiz.get(answerDto.getQuestionId()).getPoints()));
        
        yourPoints = yourPoints + plus;
        LOGGER.info(answerDto.toString());
        return new ResponseEntity<AnswerDto>(answerDto, HttpStatus.OK);
    }
\end{lstlisting}
\section*{\large\hspace*{8mm}Parametry wejściowe: id(numer pytania)\newline
\hspace*{8mm}Parametry wejściowe (JSON):}\par
\small
\begin{lstlisting}
    {
        "questionId": 1,
        "selectedAnswers": [1, 2],
        "lastQuestion": false
    }
\end{lstlisting}

\section*{\large\hspace*{8mm} Usługa put przyjmuje obiekt klasy AnswerDto zserializowany do JSON-a, który zawiera informacje od klienta, czyli id pytania,wybrane odpowiedzi zapisane w tablicy oraz informacje czy pytanie jest ostatnie.
Następnie przelicza na bieżąco punkty na podstawie odpowiedzi na pytanie udzielonych przez użytkownika. Usługa jest wywoływana zaraz po udzieleniu odpowiedzi na pytanie.
Jeżeli obiekt zostanie odebrany przez serwer poprawnie i przeliczy punkty, kod odpowiedzi serwera wyniesie 200(OK).
}\par
%PUT KONIEC
\newpage %KONIEC STRONY

%POST POCZTEK
\section*{\large\hspace*{8mm} c) Usługa POST URL(/quiz/report)
}\par
\small
\begin{lstlisting}
@PostMapping(value = "/quiz/report")
    public ResponseEntity<ArrayList<AnswerDto>> test3(@RequestBody ArrayList<AnswerDto> answerDto) {
        LOGGER.info("Tworzenie raportu:");
        FileData fileData = fileService.createFile(answerDto, PATH);
        if(QuizCode.ifPassed(QuizApiController.getYourPoints())){
            LOGGER.info("Wynik quizu jest pozytywny.");
            return new ResponseEntity<>(answerDto, HttpStatus.OK);
        }
        else{
            LOGGER.info("Wynik quizu jest negatywny.");
            return new ResponseEntity<>(HttpStatus.BAD_REQUEST);
        }
\end{lstlisting}

\section*{\large\hspace*{8mm} Usługa post przyjmuje listę obiektów AnswerDto zserializowaną do JSON-a, która zawiera wszystkie potrzebne informacje, następnie generuje raport w formacie PDF na podstawie danych zebranych podczas trwania testu  w którym wypisuje listę odebranych obiektów z informacją czy poprawnie odpowiedziano na dane pytanie, oraz z liczbą uzyskanych punktów i informacją czy test został zdany lub nie.\newline
Kody odpowiedzi serwera:\newline
200 - gdy test został zaliczony.\newline
400 - gdy nie uzyskano zaliczenia.
}\par
%POST KONIEC
%OPIS USLUG

%WYKORZYSTANE TECHNOLOGIE
\mypara Wykorzystane technologie 
\section*{\large Aplikacja jest napisana zgodnie z architekturą REST\newline\newline a)Java - język programowania w którym zostały napisane zarówno serwer jak i aplikacja klienta.\newline\newline b)Intellij IDEA - środowisko programistyczne w którym powstał nasz projekt.\newline\newline c)Spring Framework - platforma która uprościła tworzenie oprogramowania dzięki umożliwieniu wstrzykiwania zależności.\newline\newline d)Maven - narzędzie automatyzujące budowę oprogramowania.W naszym projekcie znajduje się plik pom.xml który łączy nasz projekt w Javie z Maven. W pliku zapisana jest całkowita konfiguracja związana z projektem. Zostały w nim również dodane biblioteki lombok i iText.\newline\newline e)JSON - format wymiany danych, który był nam pomocny zarówno przy przesyłaniu danych do klienta, jak i odbieraniu ich od niego.\newline\newline f)Bitbucket i Git - hostingowy serwis internetowy dzięki któremu zarządzaliśmy projektem poprzez zdalne repozytorium, pozwolił nam na współpracę w zakresie tworzenia kodu, wykorzystujący system kontroli wersji Git.\newline\newline g)Biblioteka LOMBOK - biblioteka która ułatwia pisanie kodu. Dzięki niej przy pomocy adnotacji tworzyliśmy gotowe gettery, settery itp.
}\par
%/WYKORZYSTANE TECHNOLOGIE

\newpage
%PODZIAŁ PRAC
\mypara Podział prac w zespole \newline

%do tabeli

\setlength{\arrayrulewidth}{0.6mm}
\setlength{\tabcolsep}{12pt}
\renewcommand{\arraystretch}{1.5}
\Large
\begin{tabular}{ |p{4cm}|p{12cm}| }
\hline
Gabriela Jasnosz& Sporządzenie obsługi strumieni dla pliku csv.\newline Przygotowanie usługi GET po stronie serwera.\newline Przygotowanie metod i klas zwiazanych z tworzeniem raportu oraz sprawdzaniem czy quiz został zaliczony - liczenie punktów i spradzanie odpowiedzi. \newline Przygotowanie dokumentacji w programie Word.\\
\hline

Kamil Budzik& Przygotowanie prymitywnej wersji programu na której potem bazowano.\newline Przygotowanie GUI wraz z jego metodami.\newline Konwersja dokumentacji na LATEX.\newline Przygotowanie klasy AnswerData do wysyłania odpowiedzi serwerowi.
\\
\hline

Kamil Pinas& Usługi (GET,PUT,POST) po stronie klienta i komunikacja z serwerem.\newline Przygotowanie odpowiednich klas by mogło to zostać zrealizowane.\newline Pomoc w pisaniu dokumentacji. \\
\hline
%koniec tabeli
\end{tabular}
%/PODZIAŁ PRAC

\end{document}